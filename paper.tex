\documentclass[12pt]{article}
\usepackage{graphicx} % Required for inserting images
\usepackage{biblatex}
\usepackage{hyperref}
\usepackage{fullpage}
\usepackage{libertinus}

\bibliography{rust_model_checking}

\title{CS6110: Software Verification Final Project: Formal Verification of Rust Programs with Creusot}
\author{Taylor Allred (taylor.allred21@gmail.com)\\ Ashton Wiersdorf (academics@wiersdorfmail.net)}
\date{Spring 2023}

\begin{document}

\maketitle

\begin{abstract}
\noindent
Rust is the hottest new language in systems programming.
Its linear type system provides compile-time guarantees about memory resources and avoids race-conditions.
With an increased reliance on Rust for critical code, there's a greater need for verification tools.
Several new tools have popped up to fill this need: in particular, we are looking at Creusot, a formal verification system, and Kani, a model checker.
In our project we will be verifying code with Creusot and/or Kani.
\end{abstract}

\section{Overview of Creusot}

% Have a diagram or something of how all the pieces fit together

\subsection{Understanding Why3}

% Include discussion of how why3 differs from e.g. Coq (proof automator vs. proof assistant)

\subsection{SAT Solvers}

% Brief discussion on what e.g. z3 &co. are

\section{How Rust enables Creusot}

% How does Creusot take advantage of Rust's unique type system & stuff to work?

% Why is it cool that it's using Rust?
% - macros
% - in-line documentation/verification markup

\section{Verification with Creusot}

\subsection{Example: verifying a very simple function}

\subsection{Example: a more complex example}

\section{Experience Report: verifying a library} % ← maybe; big maybe

\section{Conclusion}

% Discuss the merits of using Creusot---maybe include some UI/UX criticisms

\section{Resources}

\begin{description}
  \item[Creusot] \cite{denisCreusot2023} \cite{denisCreusotFoundryDeductive2022}

    Creusot is a formal verification tool written in Rust for verifying Rust programs.
    
  \item[CreuSAT] \cite{skotamCreuSAT2023}

    CreuSAT is a SAT solver verified with Creusot.
    
  \item[Kani] \cite{Kani2023}

    From the README: Kani is particularly useful for verifying unsafe code in
    Rust, where many of the language's usual guarantees are no longer checked by
    the compiler.

\end{description}

\printbibliography

\end{document}

% Local Variables:
% jinx-local-words: "Kani"
% End:
